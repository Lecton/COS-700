\asection{Introduction}
\label{Introduction}
\subsection{Introduction}
Graphs are used in a variety of disciplines and they serve a multitude of purposes, such as rep-
resenting the syntactical representations of the components in a sentence and their relationships
to each other[17], describing the structures of chemical compounds in chemistry[3], representing
the entries in a database and their relationships, pattern recognition and computer vision[5]. Thus
because the digraph database structure is so widely used, it is important to find methods of finding
the correlation between of data schemas in applications that make use of this comprehensive data
structure[4]. This process is referred to as graph matching. Before we get to some of the graph
matching algorithms, some concepts of graph theory will need to be understood, and they are discussed in Section \ref{Graph Overview}.

\subsection{Problem Statement}
Evaluating the isomorphic (syntactic) relationship between two or more digraphs
using search algorithms, and then comparing the similarities of the graphs is an expensive enterprise. 
This is because the procedures that are required to achieve this task are often expensive.
Thus cases where the degree of similarity between two graphs is evaluated,
only to find that the graphs are not syntactically similar after a large amount of time and resources
have been exhausted is very common, particular in cases where the graphs contain large datasets.\newline\newline
Graph matching algorithms such as the VF2 algorithm[5] compare the syntactic relationship between graphs by iterating through the graphs
from one node to another, and evaluating once node at a time in order to perform similarity relation comparison, and because the comparison
operation requires processing of all the nodes, the larger the graphs are the more time and resources must be used in order to completely evaluate this
relation between the graphs. Thus better search algorithms and strategies are required in order to
improve the efficiency of the comparison.

\subsection{Objective}
In order to improve on the quality of the comparison between digraphs, efficient search algorithms
that perform comprehensive comparison on the syntactical relationships between
graphs as well as their subgraphs are investigate in order to perform comprehensive graph matching
relative the generic brute force approach of simply iterative through each graph node and comparing its
relative position. The graph search algorithms that are required and thus investigated must be
efficient in terms of their respective space and time complexities.\newline\newline
Apart from being efficient, the algorithms must also perform graph matching on lower levels of
granularity of the graphs, thus the algorithms must also compare the syntactic comparisons on the subgraphs
for all the permutations of the two graphs subgraphs, so that the optimal graph matched result can be generated.
