\section{Experimentation}
\label{Experimentation}

\subimport{data/}{data}

\subsection{Experiment}
The experiment that is conducted on the Ullman [1] and the VF2 [11] algorithm uses a quantitative approach that evaluates the algorithms against each 
other. The search graph matching algorithms are implemented and experimented on, using two digraph or empty graph objects at a time. The experiment for that is 
employed, measured and records the algorithms ability and performance of the algorithms in cases where Graph $G${\tiny A} and Graph $G${\tiny B} are 
either partially or completely isomorphic.\newline\newline
Several experiments are done on the two algorithms so that we can understand their behaviour in different conditions, and these experiments that are conducted 
are done on both joint and disjoint graphs. The experiments that are performed are expained below.

\subsubsection{The comparison criteria}
\label{The comparison criteria}
The evaluation performed on the algorithm results is based on two criteria's. These are use to weigh the algorithms against against each other to determine which amongst them is better. The criteria's are listed below.\newline\newline
\textbf{Space efficiency:} This factor measures the amount of virtual memory (RAM) that is used by the algorithm in the graph matching process of its
execution.The unit used to measure the amount of memory  by the respective algorithms is measured in \textit{bytes (b)}. \newline\newline
\textbf{Time efficiency:} This factor measures the time taken by the algorithm to start and \textit{successfully} complete the graph matching process of its
execution. The unit used to measure the amount of time taken by the respective algorithms is measured in \textit{millisecond (ms)}. 

\subsubsection{Description Conducted Experiments}
\label{Description Conducted Experiments}
This section lists each of the experiments that where conducted for both algorithms as well as provides an explains of each experiment.
\paragraph{Performance evaluation for different vertice numbers}\mbox{}\\
\label{Performance evaluation for different vertice numbers}
This experiment uses data that is mentioned in chapter \ref{Data Set}. The experiment is conducted on joint and disjoint graphs. This experiment is intended on evaluating the behaviour of the two algorithms performance for graphs of 
different vertice numbers. The number of the vertices in the graphs range from 55 to 250 vertices in the graphs.

\paragraph{Performance evaluation for different edge numbers}\mbox{}\\
This experiment uses generated graph data comprising of 1000 vertices and 499500 edges. The generated graph data is defined using the following rules.
\begin{myEnumerate}
  \item Define a set of vertices \textit{V} for some graph \textit{G}.
  \item Each vertice \textit{v} in \textit{V} is connect to every other vertice in \textit{V}
  \item Non of the edges in the graph G are reflexive.
\end{myEnumerate}
This experiment is intended on evaluating the behaviour of the algorithms for when they are searching for subgraphs of various sizes proprotional to some 
 supergraph $G${\tiny SUP}.\newline
The subgraphs are generated from the supergraph $G${\tiny SUP} for different percentages e.g. a generated subgraph that is 10\% of the supergraph $G${\tiny SUP} graph.

\subsubsection{Experiment implementation}
The experimentation process is accomplished in phases and each phase is explained below. We will refer to Graphs $G${\tiny A} and $G${\tiny B} in this context.
\paragraph{Generate two graphs, $G${\tiny A} and $G${\tiny B}}\mbox{}\\
In the first phase of the experimentation process, a supergraph $G${\tiny A} and a subgraph $G${\tiny B} are generated. The two graphs are built together 
with their associate vertices and edges.
\paragraph{Syntactic comparison}\mbox{}\\
The comparison that is done, is dependent on the relationship between Graph $G${\tiny A} and $G${\tiny B}.
\begin{myEnumerate}
\item $G${\tiny A} is compared with the whole of $G${\tiny B} for syntactical similarity.										
\begin{myEnumerate}
\item A new subgraph of Graph B is generated.
\item The subgraph is compared with Graph $G${\tiny A} for syntactical similarity.											
\begin{myEnumerate}
\item The time taken to perform the comparison is recorded.
\item The amount of memory used by the algorithm to perform the comparison is recorded.
\end{myEnumerate}
\end{myEnumerate}
\item If Graph $G${\tiny A} and $G${\tiny B} are not completely syntactically similar, Graph $G${\tiny A} is compared with all the possible subgraphs of Graph B.
\end{myEnumerate}

\subsubsection{Result presevation}
The amount of time and memory required by both alogrithm to complete the comparison procedure of each respective experiment is stored and graphically represented. 
The results from the algorithms are weighed against a comparison criteria, and it is this criteria that is used to evaluate the algorithms against each other.\newline\newline
Once the results have been obtained and throughly evaluated, then the algorithms with the best perform per criteria field are recorded.\newline\newline
And from that set, the best algorithm is chosen overall relative to the others based on the criteria.\newpage